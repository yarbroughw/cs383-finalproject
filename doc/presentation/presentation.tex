\documentclass{beamer}
\usepackage{minted}
\usepackage{amsmath}
\usemintedstyle{trac}

\title{\huge Boids!}
\author[Weisman and Yarbrough]{Hawk Weisman and Willem Yarbrough}
\institute[Allegheny College]{Department of Computer Science \\ Allegheny College}
\date{\today}

\beamertemplatenavigationsymbolsempty

\begin{document}

\begin{frame}
  \titlepage
\end{frame}

\begin{frame}
    \huge{What are Boids?}\normalsize
    \begin{itemize}
        \item An artificial life simulation~\cite{hartman2006autonomous,reynolds1987flocks}
        \item `Bird-oid' flocking behaviour~\cite{hartman2006autonomous,reynolds1987flocks}
        \item First described by Craig Reynolds in 1987~\cite{reynolds1987flocks}
    \end{itemize}
\end{frame}

\begin{frame}
    \huge{Why Boids?}\normalsize
    \begin{itemize}
        \item Some major appearances:
        \begin{itemize}
            \item \textit{Half-Life} (1998)
            \item \textit{Batman Returns} (1992)
        \end{itemize}
        \item Other applications:
        \begin{itemize}
            \item Swarm optimization
            \item Unmanned vehicle guidance
        \end{itemize}
    \end{itemize}
\end{frame}

\begin{frame}
\huge{Our Implementation}\normalsize
\begin{itemize}
    \item \textbf{Simulation}: Boids in a toroidal 2D space
    \item \textbf{Haskell} programming language:
    \begin{itemize}
        \item A strongly-typed, lazy, purely functional programming language
        \item Why Haskell?
        \begin{itemize}
            \item Good for rapid prototyping~\cite{hudak1994haskell}
            \item Modularity~\cite{hughes1989functional}
            \item Prior experience
            \item Explore non-OO ways of representing agents
        \end{itemize}
    \end{itemize}
\end{itemize}

\end{frame}

\begin{frame}[fragile]
    \huge{What is a Boid?}\normalsize
    \begin{itemize}
        \item A boid consists of:
        \begin{itemize}
            \item A position $p_i$
            \item A velocity vector $\vec{v}_i$
            \item A sight radius $r$
        \end{itemize}
        \item<2-> In Haskell:
        \begin{minted}[fontsize=\footnotesize]{haskell}
type Vector = V2 Float
type Point  = V2 Float
type Radius = Float

data Boid = Boid { position :: !Point
                 , velocity :: !Vector
                 , radius   :: !Radius
                 }
  deriving (Show)
        \end{minted}
    \end{itemize}
\end{frame}

\begin{frame}[fragile]
    \huge{Boid Behaviour}\normalsize
    \begin{itemize}
        \item First, we define some types:
        \begin{minted}[fontsize=\footnotesize]{haskell}
type Update     = Boid -> Boid
type Perception = [Boid]
type Behaviour  = Perception -> Update
        \end{minted}
        \item<2-> Functions for finding a boid's neighborhood:
        \begin{minted}[fontsize=\footnotesize]{haskell}
inCircle :: Point -> Radius -> Point -> Bool
inCircle p_0 r p_i = ((x_i - x)^n + (y_i - y)^n) <= r^n
  where x_i = p_i ^._x
        y_i = p_i ^._y
        x   = p_0 ^._x
        y   = p_0 ^._y
        n   = 2 :: Integer

neighborhood :: World -> Boid -> Perception
neighborhood world self =
    filter (inCircle cent rad . position) world
    where cent = position self
          rad  = radius self
    \end{minted}
    \end{itemize}
\end{frame}

\begin{frame}[fragile]
    \huge{Separation steering vector}\normalsize
    \begin{itemize}

    \item Tendency to avoid collisions with other boids
        \begin{equation*}
    \vec{s}_i = - \sum\limits_{\forall b_j \in V_i} (p_i - p_j)
    \end{equation*}
    \item<2-> In Haskell:
    \begin{minted}[fontsize=\footnotesize]{haskell}
separation :: Boid -> Perception -> Vector
separation self neighbors =
    let p = position self
    in negated $
         sumV $ map (^-^ p) $ positions neighbors
\end{minted}
    \end{itemize}


\end{frame}

\begin{frame}[fragile]
\huge{Cohesion steering vector}\normalsize
    \begin{itemize}
        \item Tendency to steer towards the centre of visible boids
        \item Calculated in two steps.
        \item<2-> \textbf{Step I}: Find the centre:
            \begin{equation*}
            c_i = \sum\limits_{\forall b_j \in V_i} \frac{p_j}{m}
            \end{equation*}
        \item<3> In Haskell:
        \begin{minted}[fontsize=\footnotesize]{haskell}
centre :: Perception -> Vector
centre boids =
    let m = fromIntegral $ length boids :: Float
    in sumV (positions boids) ^/ m
        \end{minted}
    \end{itemize}

\end{frame}

\begin{frame}[fragile]
\huge{Cohesion steering vector}\normalsize
    \begin{itemize}
        \item Tendency to steer towards the centre of visible boids
        \item Calculated in two steps.
\item<1-> \textbf{Step II}: Find the cohesion vector:
            \begin{equation*}
            c_i = \sum\limits_{\forall b_j \in V_i} \frac{p_j}{m}
            \end{equation*}
        \item<2> In Haskell:
        \begin{minted}[fontsize=\footnotesize]{haskell}
cohesion :: Boid -> Perception -> Vector
cohesion self neighbors =
    let p = position self
    in centre neighbors ^-^ p
        \end{minted}
    \end{itemize}
\end{frame}

\begin{frame}[fragile]
    \huge{Alignment steering vector}\normalsize
    \begin{itemize}
    \item Tendency to match velocity with visible boids
    \begin{equation*}
    \vec{m}_i = \sum\limits_{\forall b_j \in V_i} \frac{\vec{v}_j}{m}
    \end{equation*}
    \item<2-> In Haskell:
    \begin{minted}[fontsize=\footnotesize]{haskell}
alignment :: Boid -> Perception -> Vector
       -- :: Boid -> [Boid]     -> V2 Float
alignment _ []        = V2 0 0
alignment _ neighbors =
    let m = fromIntegral $ length neighbors :: Float
    in (sumV $ map velocity neighbors) ^/ m
    \end{minted}
    \end{itemize}
\end{frame}


\begin{frame}[fragile]
    \huge{Simulating a boid}\normalsize
    \begin{enumerate}
    \item<1-> Velocity update
    \begin{equation*}
    \vec{v_i\prime} = \vec{v_i} + S.\vec{s_i} + K.\vec{k_i} + M.\vec{m_i}
    \end{equation*}
        Where $S$, $K$, and $M \in [0,1]$
    \item<2-> Position update
    \begin{equation*}
    p\prime_i = p_i + \Delta t\vec{v_i}
    \end{equation*}
    \end{enumerate}
\end{frame}

\begin{frame}[fragile]
    \huge{Simulating a boid}\normalsize
    \begin{itemize}
    \item In Haskell:
        \begin{minted}{haskell}
steer :: Weights -> Behaviour
steer (s, c, m) neighbors self =
    let s_i  = s *^ separation self neighbors
        c_i  = c *^ cohesion self neighbors
        m_i  = m *^ alignment self neighbors
        v'   = velocity self ^+^ s_i ^+^ c_i ^+^ m_i
        p    = position self
        p'   = p ^+^ v'
    in self { position = p', velocity = v'}
        \end{minted}
    \end{itemize}
\end{frame}

\begin{frame}
    \huge{A brief demonstration}\normalsize
\end{frame}

\begin{frame}
\huge{References}\normalsize
\footnotesize{

 \bibliographystyle{plain}
 \bibliography{../assets/references}
}
\end{frame}

\end{document}