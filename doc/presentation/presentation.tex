\documentclass{beamer}
\usepackage{minted}
\usepackage{amsmath}
\usemintedstyle{trac}

\title{\huge Boids!}
\author[Weisman and Yarbrough]{Hawk Weisman and Willem Yarbrough}
\institute[Allegheny College]{Department of Computer Science \\ Allegheny College}
\date{\today}

\begin{document}

\begin{frame}
  \titlepage
\end{frame}

\begin{frame}
    \huge{What are Boids?}\normalsize
    \begin{itemize}
        \item An artificial life simulation~\cite{hartman2006autonomous,reynolds1987flocks}
        \item `Bird-oid' flocking behaviour~\cite{hartman2006autonomous,reynolds1987flocks}
        \item first described by Craig Reynolds in 1987~\cite{reynolds1987flocks}
    \end{itemize}
\end{frame}

\begin{frame}
    \huge{Why Boids?}\normalsize
    \begin{itemize}
        \item Some major appearances:
        \begin{itemize}
            \item \textit{Half-Life} (1998)
            \item \textit{Batman Returns} (1992)
        \end{itemize}
        \item Other applications:
        \begin{itemize}
            \item Swarm optimization
            \item Unmanned vehicle guidance
        \end{itemize}
    \end{itemize}
\end{frame}

\begin{frame}
\huge{Our Implementation}\normalsize
\begin{itemize}
    \item \textbf{Haskell} programming language:
    \begin{itemize}
        \item It's good~\cite{hudak1994haskell}
        \item Will should explain this; it's his pet toy language
    \end{itemize}
\end{itemize}

\end{frame}

\begin{frame}[fragile]
    \huge{What is a Boid?}\normalsize
    \begin{itemize}
        \item Consists of
        \begin{itemize}
            \item A position $p_i$
            \item A velocity vector $\vec{v}_i$
            \item A sight radius $r$
        \end{itemize}
        \item In Haskell:
        \begin{minted}[fontsize=\footnotesize]{haskell}
type Vector = V2 Float
type Point  = V2 Float
type Radius = Float

data Boid = Boid { position :: !Point
                 , velocity :: !Vector
                 , radius   :: !Radius
                 }
  deriving (Show)
        \end{minted}
    \end{itemize}
\end{frame}

\begin{frame}[fragile]
    \huge{Separation steering vector}\normalsize
    \begin{itemize}

    \item Tendency to avoid collisions with other boids
        \begin{equation*}
    \vec{s}_i = - \sum\limits_{\forall b_j \in V_i} (p_i - p_j)
    \end{equation*}
    \item In Haskell:
    \begin{minted}[fontsize=\footnotesize]{haskell}
separation :: Boid -> Perception -> Vector
separation self neighbors =
    let p = position self
    in negated $
         sumV $ map (^-^ p) $ positions neighbors
\end{minted}
    \end{itemize}


\end{frame}

\begin{frame}[fragile]
\huge{Cohesion steering vector}\normalsize
    \begin{itemize}
        \item Tendency to steer towards the centre of visible boids
        \item Calculated in two steps:
                \begin{equation}
        c_i = \sum\limits_{\forall b_j \in V_i} \frac{p_j}{m}
        \end{equation}
        \begin{equation}
        \vec{k}_i = c_i - p_i
        \end{equation}
        \item In Haskell:
        \begin{minted}[fontsize=\footnotesize]{haskell}
centre :: Perception -> Vector
centre boids =
    let m = fromIntegral $ length boids :: Float
    in sumV (positions boids) ^/ m
cohesion :: Boid -> Perception -> Vector
cohesion self neighbors =
    let p = position self
    in centre neighbors ^-^ p
        \end{minted}

    \end{itemize}


\end{frame}

\begin{frame}[fragile]
    \huge{Alignment steering vector}\normalsize
    \begin{itemize}
    \item Tendency to match velocity with visible boids
    \begin{equation*}
    \vec{m}_i = \sum\limits_{\forall b_j \in V_i} \frac{\vec{v}_j}{m}
    \end{equation*}
    \item In Haskell:
    \begin{minted}[fontsize=\footnotesize]{haskell}
alignment :: Boid -> Perception -> Vector
alignment _ []        = V2 0 0
alignment _ neighbors =
    let m = fromIntegral $ length neighbors :: Float
    in (sumV $ map velocity neighbors) ^/ m
    \end{minted}
    \end{itemize}

\end{frame}
\begin{frame}
\huge{References}\normalsize
 \bibliographystyle{plain}
 \bibliography{../assets/references}
\end{frame}

\end{document}